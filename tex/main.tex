\documentclass[onecolumn, compsoc,10pt]{IEEEtran}
\let\labelindent\relax
\usepackage{enumitem}
\usepackage{etex}
\newcommand\hmmax{0}
\newcommand\bmmax{0}
\usepackage{amssymb,amsfonts,amsmath,amsthm}
\usepackage{graphicx}
\usepackage[usenames,x11names, dvipsnames, svgnames]{xcolor}
\usepackage{amsmath,amssymb}
\usepackage{dsfont}
\usepackage{amsfonts}
\usepackage{mathrsfs}
\usepackage{texshade}
\usepackage{hyperref}
\hypersetup{
  colorlinks=true,
  linkcolor=black,
  citecolor=black,
  filecolor=black,
  urlcolor=DodgerBlue4,
  breaklinks=false,
  % linkbordercolor=red,% hyperlink borders will be red
  % pdfborderstyle={/S/U/W 1}% border style will be underline of width 1pt
}
\usepackage{array}
\usepackage{xr}
\usepackage{verbatim}
\usepackage{multirow}
\usepackage{longtable}
\usepackage{siunitx}
\usepackage{booktabs}
\usepackage{tabularx}
\usepackage{ragged2e}

\usepackage{bm}
% \usepackage[T1,euler-digits]{eulervm}
% \usepackage{times}
% \usepackage{pxfonts}
\usepackage{tikz}
\usepackage{pgfplots}
\usetikzlibrary{shapes,calc,shadows,fadings,arrows,decorations.pathreplacing,automata,positioning}
\usetikzlibrary{external}
\usetikzlibrary{decorations.text}
\usepgfplotslibrary{colorbrewer} 
\usepgfplotslibrary{statistics}
\usetikzlibrary{shapes,calc,shadows,fadings,arrows,decorations.pathreplacing,automata,positioning}
\usetikzlibrary{shadows.blur}
\usetikzlibrary{fit}
\usepackage{comment}
\usepackage[size=normal]{subcaption}

\tikzexternalize[prefix=./Figures/External/]% activate externalization!
\tikzexternaldisable
% \addtolength{\voffset}{.1in}  
\usepackage{geometry}
\geometry{a4paper, left=.65in,right=.65in,top=.8in,bottom=0.7in}

\usepackage{pdflscape}

\addtolength{\textwidth}{-.1in}    
\addtolength{\hoffset}{.05in}    
\addtolength{\textheight}{.1in}    
\addtolength{\footskip}{0in}    
\usepackage{rotating}
\definecolor{nodecol}{RGB}{240,240,220}
\definecolor{nodeedge}{RGB}{240,240,225}
\definecolor{edgecol}{RGB}{130,130,130}
\tikzset{%
  fshadow/.style={      preaction={
      fill=black,opacity=.3,
      path fading=circle with fuzzy edge 20 percent,
      transform canvas={xshift=1mm,yshift=-1mm}
    }} 
}
\usetikzlibrary{pgfplots.dateplot}
\usetikzlibrary{patterns}
\usetikzlibrary{decorations.markings}
\usepackage{fancyhdr}
\usepackage{mathtools}
\usepackage{datetime}
\usepackage{comment}
%% ## Equation Space Control---------------------------
\def\EQSP{3pt}
\newcommand{\mltlne}[2][\EQSP]{\begingroup\setlength\abovedisplayskip{#1}\setlength\belowdisplayskip{#1}\begin{equation}\begin{multlined} #2 \end{multlined}\end{equation}\endgroup\noindent}
\newcommand{\cgather}[2][\EQSP]{\begingroup\setlength\abovedisplayskip{#1}\setlength\belowdisplayskip{#1}\begin{gather} #2 \end{gather}\endgroup\noindent}
\newcommand{\cgathers}[2][\EQSP]{\begingroup\setlength\abovedisplayskip{#1}\setlength\belowdisplayskip{#1}\begin{gather*} #2 \end{gather*}\endgroup\noindent}
\newcommand{\calign}[2][\EQSP]{\begingroup\setlength\abovedisplayskip{#1}\setlength\belowdisplayskip{#1}\begin{align} #2 \end{align}\endgroup\noindent}
\newcommand{\caligns}[2][\EQSP]{\begingroup\setlength\abovedisplayskip{#1}\setlength\belowdisplayskip{#1}\begin{align*} #2 \end{align*}\endgroup\noindent}
\newcommand{\mnp}[2]{\begin{minipage}{#1}#2\end{minipage}} 
%% COLOR DEFS------------------------------------------
\newtheorem{thm}{Theorem}
\newtheorem{cor}{Corollary}
\newtheorem{lem}{Lemma}
\newtheorem{prop}{Proposition}
\newtheorem{defn}{Definition}
\newtheorem{exmpl}{Example}
\newtheorem{rem}{Remark}
\newtheorem{notn}{Notation}
%% ------------PROOF INCLUSION -----------------
\def\NOPROOF{Proof omitted.}
\newif\ifproof
\prooffalse % or \draftfalse
\newcommand{\Proof}[1]{
  \ifproof
  \begin{IEEEproof}
    #1\end{IEEEproof}
  \else
  \NOPROOF
  \fi
}
%% ------------ -----------------
\newcommand{\DETAILS}[1]{#1}
%% ------------ -----------------
% color commands------------------------
\newcommand{\etal}{\textit{et} \mspace{3mu} \textit{al.}}
% \renewcommand{\algorithmiccomment}[1]{$/** $ #1 $ **/$}
\newcommand{\vect}[1]{\textbf{\textit{#1}}}
\newcommand{\figfont}{\fontsize{8}{8}\selectfont\strut}
\newcommand{\hlt}{ \bf \sffamily \itshape\color[rgb]{.1,.2,.45}}
\newcommand{\pitilde}{\widetilde{\pi}}
\newcommand{\Pitilde}{\widetilde{\Pi}}
\newcommand{\bvec}{\vartheta}
\newcommand{\algo}{\textrm{\bf\texttt{GenESeSS}}\xspace}
\newcommand{\xalgo}{\textrm{\bf\texttt{xGenESeSS}}\xspace}
\newcommand{\FNTST}{\bf }
\newcommand{\FNTED}{\color{darkgray} \scriptsize $\phantom{.}$}
\renewcommand{\baselinestretch}{.95}
\newcommand{\sync}{\otimes}
\newcommand{\psync}{\hspace{3pt}\overrightarrow{\hspace{-3pt}\sync}}
% \newcommand{\psync}{\raisebox{-4pt}{\begin{tikzpicture}\node[anchor=south] (A) {$\sync$};
%   \draw [->,>=stealth] ([yshift=-2pt, xshift=2pt]A.north west) -- ([yshift=-2pt]A.north east); %\end{tikzpicture}}}
\newcommand{\base}[1]{\llbracket #1 \rrbracket}
\newcommand{\nst}{\textrm{\sffamily\textsc{Numstates}}}
\newcommand{\HA}{\boldsymbol{\mathds{H}}}
\newcommand{\eqp}{ \vartheta }
\newcommand{\entropy}[1]{\boldsymbol{h}\left ( #1 \right )}
\newcommand{\norm}[1]{\left\lVert #1 \right\rVert}%
\newcommand{\abs}[1]{\left\lvert #1 \right\rvert}%
\newcommand{\absB}[1]{\big\lvert #1 \big\rvert}%
% #############################################################
% #############################################################
% PREAMBLE ####################################################
% #############################################################
% #############################################################
% \usepackage{pnastwoF}      
\DeclareMathOperator*{\argmax}{argmax}
\DeclareMathOperator*{\argmin}{arg\,min}
\DeclareMathOperator*{\expect}{\mathbf{E}}
\DeclareMathOperator*{\var}{\mathbf{Var}}

\newcommand{\ND}{ \mathcal{N}  }
\usepackage[linesnumbered,ruled,vlined,noend]{algorithm2e}
\newcommand{\captionN}[1]{\caption{\color{darkgray} \sffamily \fontsize{9}{10}\selectfont #1  }}
\newcommand{\btl}{\ \textbf{\small\sffamily bits/letter}}
\usepackage{txfonts}
% \usepackage{ccfonts}
%%% save defaults
\renewcommand{\rmdefault}{phv} % Arial
\renewcommand{\sfdefault}{phv} % Arial
\edef\keptrmdefault{\rmdefault}
\edef\keptsfdefault{\sfdefault}
\edef\keptttdefault{\ttdefault}

% \usepackage{kerkis}
\usepackage[OT1]{fontenc}
\usepackage{concmath}
% \usepackage[T1]{eulervm} 
% \usepackage[OT1]{fontenc}
%%% restore defaults
\edef\rmdefault{\keptrmdefault}
\edef\sfdefault{\keptsfdefault}
\edef\ttdefault{\keptttdefault}
\tikzexternalenable
% ##########################################################
\tikzfading[name=fade out,
inner color=transparent!0,
outer color=transparent!100]
% ###################################
\newcommand{\xtitaut}[2]{
  \noindent\mnp{\textwidth}{
    \mnp{\textwidth}{\raggedright\Huge \bf \sffamily #1}

    \vskip 1em

    {\bf \sffamily #2}
  }
  \vskip 2em
}
% ###################################
% ###################################
\tikzset{wiggle/.style={decorate, decoration={random steps, amplitude=10pt}}}
\usetikzlibrary{decorations.pathmorphing}
\pgfdeclaredecoration{Snake}{initial}
{
  \state{initial}[switch if less than=+.625\pgfdecorationsegmentlength to final,
  width=+.3125\pgfdecorationsegmentlength,
  next state=down]{
    \pgfpathmoveto{\pgfqpoint{0pt}{\pgfdecorationsegmentamplitude}}
  }
  \state{down}[switch if less than=+.8125\pgfdecorationsegmentlength to end down,
  width=+.5\pgfdecorationsegmentlength,
  next state=up]{
    \pgfpathcosine{\pgfqpoint{.25\pgfdecorationsegmentlength}{-1\pgfdecorationsegmentamplitude}}
    \pgfpathsine{\pgfqpoint{.25\pgfdecorationsegmentlength}{-1\pgfdecorationsegmentamplitude}}
  }
  \state{up}[switch if less than=+.8125\pgfdecorationsegmentlength to end up,
  width=+.5\pgfdecorationsegmentlength,
  next state=down]{
    \pgfpathcosine{\pgfqpoint{.25\pgfdecorationsegmentlength}{\pgfdecorationsegmentamplitude}}
    \pgfpathsine{\pgfqpoint{.25\pgfdecorationsegmentlength}{\pgfdecorationsegmentamplitude}}
  }
  \state{end down}[width=+.3125\pgfdecorationsegmentlength,
  next state=final]{
    \pgfpathcosine{\pgfqpoint{.15625\pgfdecorationsegmentlength}{-.5\pgfdecorationsegmentamplitude}}
    \pgfpathsine{\pgfqpoint{.15625\pgfdecorationsegmentlength}{-.5\pgfdecorationsegmentamplitude}}
  }
  \state{end up}[width=+.3125\pgfdecorationsegmentlength,
  next state=final]{
    \pgfpathcosine{\pgfqpoint{.15625\pgfdecorationsegmentlength}{.5\pgfdecorationsegmentamplitude}}
    \pgfpathsine{\pgfqpoint{.15625\pgfdecorationsegmentlength}{.5\pgfdecorationsegmentamplitude}}
  }
  \state{final}{\pgfpathlineto{\pgfpointdecoratedpathlast}}
}
% ###################################
% ###################################
\newcolumntype{L}[1]{>{\rule{0pt}{2ex}\raggedright\let\newline\\\arraybackslash\hspace{0pt}}m{#1}}
\newcolumntype{C}[1]{>{\rule{0pt}{2ex}\centering\let\newline\\\arraybackslash\hspace{0pt}}m{#1}}
\newcolumntype{R}[1]{>{\rule{0pt}{2ex}\raggedleft\let\newline\\\arraybackslash\hspace{0pt}}m{#1}}



% ################################################
% ################################################
% ################################################
% ################################################
% ####################################
\newcommand{\set}[1]{\left\{ #1 \right\}}
\newcommand{\paren}[1]{\left( #1 \right)}
\newcommand{\bracket}[1]{\left[ #1 \right]}
% \newcommand{\norm}[1]{\left\Vert #1 \right\Vert}
\newcommand{\nrm}[1]{\left\llbracket{#1}\right\rrbracket}
\newcommand{\parenBar}[2]{\paren{#1\,{\left\Vert\,#2\right.}}}
\newcommand{\parenBarl}[2]{\paren{\left.#1\,\right\Vert\,#2}}
\newcommand{\ie}{$i.e.$\xspace}
\newcommand{\addcitation}{\textcolor{black!50!red}{\textbf{ADD CITATION}}}
\newcommand{\subtochange}[1]{{\color{black!50!green}{#1}}}
\newcommand{\tobecompleted}{{\color{black!50!red}TO BE COMPLETED.}}


\newcommand{\pIn}{\mathscr{P}_{\textrm{in}}}
\newcommand{\pOut}{\mathscr{P}_{\textrm{out}}}
\newcommand{\aIn}[1][\Sigma]{#1_{\textrm{in}}}
\newcommand{\aOut}[1][\Sigma]{#1_{\textrm{out}}}
\newcommand{\xin}[1]{#1_{\textrm{in}}}
\newcommand{\xout}[1]{#1_{\textrm{out}}}

\newcommand{\R}{\mathbb{R}} % Set of real numbers
\newcommand{\F}[1][]{\mathcal{F}_{#1}}
\newcommand{\SR}{\mathcal{S}} % Semiring of sets
\newcommand{\RR}{\mathcal{R}} % Ring of sets
\newcommand{\N}{\mathbb{N}} % Set of natural numbers (0 included)


\newcommand{\Pp}[1][n]{\mathscr{P}^+_{#1}}
\renewcommand{\entropy}[1]{\boldsymbol{h}\left ( #1 \right )}

\def\TPR{\textrm{TPR}\xspace}
\def\TNR{\textrm{TNR}\xspace}
\def\FPR{\textrm{FPR}\xspace}
\def\PPV{\textrm{PPV}\xspace}

\usetikzlibrary{arrows.meta}
\usetikzlibrary{decorations.pathreplacing,shapes.misc}
\usepgfplotslibrary{fillbetween}
%usepackage{tikz-network}
\usetikzlibrary{shapes.geometric}
\usetikzlibrary{math}
\usepgfplotslibrary{colorbrewer} 

\usepackage{textcomp}
\usepackage{colortbl}
\usepackage{array}
\usepackage{courier} 
\usepackage{wrapfig}
\usepackage{pifont}
\usetikzlibrary{chains,backgrounds}
\usetikzlibrary{intersections}
\usetikzlibrary{pgfplots.groupplots}
\usepgfplotslibrary{fillbetween} 
\usetikzlibrary{arrows.meta}
\usepackage{pgfplotstable}
\usepackage[super,compress,sort,comma]{natbib}

\usepackage{setspace}
\usetikzlibrary{math}
\usetikzlibrary{matrix}
\usepackage{xstring}
\usepackage{xspace}
\usepackage{flushend}

\makeatletter
\renewcommand\section{\@startsection {section}{1}{\z@}%
  {-2ex \@plus -1ex \@minus -.2ex}%
  {1ex \@plus.1ex}%
  {\Large\bfseries\scshape}}
\renewcommand\subsection{\@startsection {subsection}{1}{\z@}%
  {-2ex \@plus -.25ex \@minus -.2ex}%
  {0.1ex \@plus.0ex}%
  {\fontsize{11}{10}\selectfont\bfseries\sffamily\color{black}}}
\renewcommand\subsubsection{\@startsection {subsubsection}{1}{\z@}%
  {0ex \@plus -.5ex \@minus -.2ex}%
  {0.0ex \@plus.5ex}%
  {\fontsize{9}{9}\selectfont\bfseries\itshape\sffamily\color{darkgray}}}
\renewcommand\paragraph{\@startsection {paragraph}{1}{\z@}%
  {-.2ex \@plus -.5ex \@minus -.2ex}%
  {0.0ex \@plus.5ex}%
  {\fontsize{9}{9}\selectfont\itshape\sffamily\color{darkgray}}}
       
%\renewcommand{\thesubsection}{\thesection.\arabic{subsection}}
\renewcommand{\thesubsectiondis}{\arabic{subsection}.}
\renewcommand{\thesectiondis}{\arabic{section}.}
\renewcommand{\thesection}{\arabic{section}}

\renewcommand{\thetable}{\arabic{table}}

\makeatother
\makeatletter
\pgfdeclareradialshading[tikz@ball]{ball}{\pgfqpoint{-10bp}{10bp}}{%
  color(0bp)=(tikz@ball!30!white);
  color(9bp)=(tikz@ball!75!white);
  color(18bp)=(tikz@ball!90!black);
  color(25bp)=(tikz@ball!70!black);
  color(50bp)=(black)}
\makeatother
%\newcommand{\tball}[1][CadetBlue4]{${\color{#1}\Large\boldsymbol{\blacksquare}}$}
\renewcommand{\baselinestretch}{1}
%\renewcommand{\captionN}[1]{\caption{\color{CadetBlue4!50!black} \sffamily \fontsize{9}{10}\selectfont #1  }}
\tikzexternaldisable 
\parskip=6pt
\parindent=0pt
%\newcommand{\Mark}[1]{\textsuperscript{#1}}
\pagestyle{fancy}

\newcounter{Dcounter}
\setcounter{Dcounter}{1}
\newcommand{\DQS}[1]{\marginpar{\tikzexternaldisable \tikz{\node[rounded corners=5pt,draw=none,thick,fill=black!10,font=\sffamily\fontsize{7}{8}\selectfont] {\mnp{.45in} {\color{Red3}\raggedright  \#\theDcounter.~#1}}; }}\stepcounter{Dcounter}\xspace}

\newcommand{\qn}[1][i]{\Phi_{#1}}
\newcommand{\D}[1][i]{\mathscr{D}\left ( {\Sigma_#1} \right ) }
\newcommand{\Dx}{\mathscr{D}}
\def\J{\mathds{J}}
\def\M{\omega}
\def\N{\mathds{N}}
\newcommand{\cp}[1][P]{\langle #1 \rangle}
\newcommand{\mem}[1]{\M_{#1}}


\makeatletter
\newcommand\transformxdimension[1]{
    \pgfmathparse{((#1/\pgfplots@x@veclength)+\pgfplots@data@scale@trafo@SHIFT@x)/10^\pgfplots@data@scale@trafo@EXPONENT@x}
}
\newcommand\transformydimension[1]{
    \pgfmathparse{((#1/\pgfplots@y@veclength)+\pgfplots@data@scale@trafo@SHIFT@y)/10^\pgfplots@data@scale@trafo@EXPONENT@y}
}
\makeatother

\parskip=6pt
\parindent=0pt


\pgfplotsset{
    discard if/.style 2 args={
        x filter/.code={
            \edef\tempa{\thisrow{#1}}
            \edef\tempb{#2}
            \ifx\tempa\tempb
                \def\pgfmathresult{inf}
            \fi
        }
    },
    discard if not/.style 2 args={
        x filter/.code={
            \edef\tempa{\thisrow{#1}}
            \edef\tempb{#2}
            \ifx\tempa\tempb
            \else
                \def\pgfmathresult{inf}
            \fi
        }
    }
  }

\def\commatononei#1,{#1}
\def\commatononej#1,#2,{#1#2}
\def\commatonone#1{\expandafter\commatononei#1}
\def\commatononeT#1{\expandafter\commatononej#1}
\newcommand{\Sum} [2] {#1 + #2 = \the\numexpr #1 + #2 \relax \\}

\newcounter{Ccounter}
\setcounter{Ccounter}{1}
\newcommand{\comm}[1]{{\color{gray} {\bf \sffamily Comment~\#\theCcounter.~}#1}\xspace}
\newcommand{\resp}[1]{{\color{Blue2} {\bf \sffamily Author Response~\#\theCcounter.~}#1}\stepcounter{Ccounter}\xspace}
  
\def\TITLE{[Ve]tting  [R]esponse [I]ntegrity from  \\cross-[T]alk  dependencies in [A]dversarial [S]urveys: \\ \itshape ``Who can catch a liar?''}

\title{\TITLE}

\def\x{\bm{\mathrm{x}}} 
\def\y{\bm{\mathrm{y}}}

\author{\sffamily  \fontsize{10}{12}\selectfont Robert Gibbons,       Ishanu Chattopadhyay$^{1,4,5,7\bigstar}$\\ 
\vspace{10pt}

\sffamily  \fontsize{10}{12}\selectfont
$^{1}$Department of Medicine, University of Chicago, Chicago, IL 60637, USA\\ 
$^{4}$Committee on Quantitative Methods in Social, Behavioral, and Health Sciences, University of Chicago, Chicago, IL 60637, USA\\
$^{5}$Committee on Genetics, Genomics \& Systems Biology, University of Chicago, Chicago, IL 60637, USA
\vskip 1em
$^\bigstar$To whom correspondence should be addressed: e-mail: \href{mailto:ishanu@uchicago.edu}{\texttt{ishanu@uchicago.edu}}.}
\title{\TITLE}


\def\TEXTCOL{gray}



\def\J{\mathds{J}}
\def\M{\omega}
%\newcommand{\mem}[1]{\M_{#1}}
\def\cognet{CogNet\xspace}
\def\qnet{Q-net\xspace}
\def\qdist{q-distance\xspace}
\def\qbiome{Qbiome\xspace}
\def\qsamp{q-sampling\xspace}

\def\x{\bm{\mathrm{x}}}
\def\y{\bm{\mathrm{y}}}
\def\erisk{$M_\delta$\xspace}
\def\RHO{A_\delta}
\def\bact{Bacteroidia\xspace}
\def\actn{Actinobacteria\xspace}
\def\gamm{Gammaproteobacteria\xspace}
\def\ubac{unclassified Bacteria\xspace}
\def\bacl{Bacilli\xspace}
\def\clsd{Clostridia\xspace}
\def\corb{Coriobacteriia\xspace}
\def\verru{Verrucomicrobia\xspace}
 
\usepackage{flushend}
%\externaldocument[SI-]{SI}
% \externaldocument[EXT-]{exfig}
\newif\iftikzX
\tikzXtrue
\tikzXfalse
\def\EXTENDED{Extended Data\xspace}
\def\SUPPLEMENTARY{Supplementary\xspace}
\newif\ifFIGS
\FIGSfalse  
\FIGStrue 

\def\METHODS{Online Methods\xspace}

  
\tikzexternalenable    
%\pgfplotsset{compat=1.18}

\begin{document}  
\maketitle  

{\bf \sffamily \fontsize{10}{12}\selectfont \noindent   
  {\normalfont \itshape Abstract:}

}
  
\vspace{10pt} 

\subsection*{Introduction}


The detection of lies and false information has been a topic of interest in various fields, including psychology, sociology, and criminology. One related area of focus is malingering, defined as the intentional feigning or exaggeration of symptoms for external gain in a medical setting. Malingering can significantly complicate the accurate diagnosis of mental disorders and undermine the validity of mental health evaluations.

In recent years, computational tools have been proposed to assist in the detection of malingering, utilizing advanced techniques such as machine learning algorithms, behavioral analysis, and psychological profiling~\cite{rogers2008clinical}. These tools aim to provide a more objective and efficient evaluation of suspected malingering cases.

It is important to note that while malingering detection is closely related to lie detection, it specifically pertains to the deliberate feigning of symptoms in a medical setting (Gudjonsson, 2003).

In this paper, we aim to provide a comprehensive overview of the current state of the art in malingering detection and its known countermeasures. We begin by discussing the concept of lie detection and its connection to malingering. We then examine the current limitations of computational tools used for malingering detection and suggest future directions for research in this field.


xxxx




The detection of lies and false information is a crucial aspect of various fields, including psychology, sociology, and criminology. In particular, the accurate detection of malingering, defined as the intentional feigning or exaggeration of symptoms for external gain in a medical setting, is of utmost importance. Malingering can significantly complicate the accurate diagnosis of mental disorders and undermine the validity of mental health evaluations.

Despite the existence of various computational tools to assist in the detection of malingering, such as machine learning algorithms, behavioral analysis, and psychological profiling, there is still room for improvement and innovation in this field~\cite{rogers2008clinical}.

In this paper, our goal is to contribute to the field of malingering detection by developing a new machine learning algorithm for algorithmic lie detection. We aim to address the limitations of current tools and provide a more objective and efficient evaluation of suspected malingering cases. Our proposed algorithm achieved a accuracy of [insert number with citation], outperforming existing techniques by [insert number with citation].

We begin by providing a comprehensive overview of the current state of the art in malingering detection and its known countermeasures. We then examine the limitations of existing computational tools and discuss the potential benefits of our proposed machine learning algorithm. We conclude by suggesting future directions for research in this field.



xxx

The detection of lies and false information has been approached through various methods, including physiological measures, behavioral analysis, and psychological profiling. Physiological measures include the examination of physiological responses such as changes in heart rate, blood pressure, and skin conductance, which have been linked to deception~\cite{Ganslen1990}. Behavioral analysis involves the observation and analysis of nonverbal cues and patterns of behavior, such as eye movements, posture, and gestures. Finally, psychological profiling involves the use of psychological tests, questionnaires, and interviews to examine a person's thought processes, motivations, and tendencies.

In recent years, computational tools have been developed to aid in the detection of lies and false information. These tools utilize advanced techniques such as machine learning algorithms and artificial intelligence to analyze patterns and correlations in physiological responses, behavioral data, and psychological profiles. The ultimate goal is to provide a more objective and efficient evaluation of suspected cases of deception or malingering.

However, it is important to note that the accuracy of these methods is still limited and subject to various biases and limitations. Further research and innovation in this field are necessary to improve the reliability and validity of lie detection techniques.

xxx

Various machine learning algorithms for lie detection have been developed and reported in the literature, utilizing different inputs such as physiological data, behavioral data, and psychological profiles.

In terms of physiological data, machine learning algorithms have been developed to analyze changes in heart rate, blood pressure, and skin conductance~\cite{Ganslen1990}. For example, a study by Chan and colleagues~\cite{Chan2018}  utilized a machine learning algorithm to analyze physiological responses collected from a polygsomnographic sleep study, achieving an accuracy of [insert number with citation].

Behavioral data, such as nonverbal cues and patterns of behavior, have also been utilized as inputs for machine learning algorithms~\cite{DePaulo2003}. For example, a study by Wang and colleagues~\cite{Wang2020} developed a machine learning algorithm to analyze patterns in eye movements and gestures during a lie detection task, achieving an accuracy of [insert number with citation].

Finally, psychological profiles, such as responses to psychological tests, questionnaires, and interviews, have also been utilized as inputs for machine learning algorithms in the detection of malingering~\cite{rogers2008clinical}. A study by Lee and colleagues~\cite{Lee2019} developed a machine learning algorithm to analyze responses to a self-reported symptom checklist, achieving an accuracy of [insert number with citation].

These studies demonstrate the potential for machine learning algorithms to provide a more objective and efficient evaluation of suspected cases of deception or malingering. However, it is important to note that the accuracy of these methods is still limited and subject to various biases and limitations. Further research and innovation in this field are necessary to improve the reliability and validity of lie detection techniques.


xxxxxxx



Malingering is a major challenge in the field of lie detection, as it can result in false negative results and reduce the accuracy of lie detection methods. Malingering is defined as the intentional manipulation of test results for personal gain, and can take various forms, such as the use of drugs to manipulate physiological responses, the use of countermeasures to deceive lie detection methods, and the fabrication of symptoms or behaviors.

Classical approaches to counter malingering include the use of multiple measures and the use of control questions. For example, in the polyggraph, multiple physiological measures are used to detect deception, and control questions are used to establish a baseline response. However, these approaches are often subject to countermeasures, and the effectiveness of these methods is limited.

More recent approaches to counter malingering include the use of psychological measures, such as the Structured Interview of Reported Symptoms (SIRS)~\cite{Wong2005}, and the use of machine learning algorithms to analyze patterns in language, facial expressions, and body movements (Yu et al., 2018). For example, machine learning algorithms can be trained on large datasets of labeled data to identify patterns in language, facial expressions, and body movements that are indicative of malingering.

In conclusion, malingering is a significant challenge in the field of lie detection, and a variety of methods have been developed to counteract it. While classical approaches, such as the use of multiple measures and control questions, have been used for many years, recent advances in technology have enabled the development of more sophisticated methods, such as the use of psychological measures and machine learning algorithms. Further research is needed to develop methods that are robust to malingering and to improve the accuracy of lie detection methods.

xxxx

The ability to detect lies has long been of interest to researchers from various disciplines, including psychology, sociology, and law enforcement. In recent years, the development of new technologies has enabled the scientific study of lie detection, and has led to the development of a variety of methods for detecting deception. One of the main challenges in lie detection is the issue of malingering, or the intentional manipulation of test results for personal gain. The relationship between lie detection and malingering is complex, and the effectiveness of different lie detection methods is often dependent on the presence of malingering and the methods used to counteract it.

The use of physiological measures, such as polygraphs and functional magnetic resonance imaging (fMRI), has been investigated as a means of detecting deception. However, these methods are often subject to countermeasures, such as the use of drugs or other means of manipulating physiological responses, which can reduce their accuracy. As a result, a number of computational approaches have been developed to detect deception, including machine learning algorithms and computer vision techniques.

One such approach is the use of machine learning algorithms to analyze patterns in language, facial expressions, and body movements. These algorithms have been trained on large datasets of labeled data, and have been shown to outperform traditional lie detection methods in certain cases. For example, studies have reported accuracy rates of over 90\% for machine learning algorithms that analyze patterns in facial expressions and body movements~\cite{Ekman1992,Yu2018}.

Another approach is the use of computer vision techniques to analyze eye movements, including gaze direction and pupil dilation. These techniques have been shown to be effective in detecting deception, with accuracy rates ranging from 60-80\%~\cite{Ducharme2017,Granziero2019}.

In conclusion, the field of lie detection is a rapidly evolving area of research, and a variety of methods have been developed to detect deception. However, the issue of malingering remains a significant challenge, and the effectiveness of different lie detection methods is often dependent on the presence of malingering and the methods used to counteract it. Further research is needed to develop methods that are robust to malingering and to improve the accuracy of lie detection methods.



xxxx


A traditional polyggraph, also known as a lie detector, is a tool that measures physiological responses to determine whether an individual is telling the truth or lying. The polyggraph typically measures several physiological responses simultaneously, including heart rate, respiration rate, blood pressure, and skin conductance. These responses are thought to reflect an individual's level of anxiety or arousal, which can be an indicator of deception.

The most common physiological measures used in a traditional polyggraph include:

Cardiovascular measures: Heart rate and blood pressure are often used as indicators of arousal, as these physiological responses can increase in response to stressful or emotionally charged situations. For example, Eckman and Friesen~\cite{Eckman1969} found that blood pressure increases in response to deception.

Respiratory measures: Respiration rate is another physiological response that is often measured during a polyggraph examination. This measure is thought to reflect an individual's level of anxiety, as breathing can become more shallow and rapid in response to stressful or emotionally charged situations.

Electrodermal measures: Skin conductance, also known as galvanic skin response (GSR), is a measure of the electrical conductance of the skin, which is thought to reflect an individual's level of arousal or emotional state. For example, Lykken~\cite{Lykken1959} found that skin conductance increases in response to deception.

The polyggraph measures these physiological responses during a controlled questioning process, in which individuals are asked a series of questions that are designed to elicit a physiological response. The responses to these questions are then compared to responses to control questions, which are designed to establish a baseline response. Based on the comparison of these responses, a conclusion is drawn as to whether the individual is telling the truth or lying.



% \subsection*{Conclusion}

% \clearpage

% \section*{\METHODS}
% %\allowdisplaybreaks{
We briefly describe the proposed computational framework. 


\subsection*{\tnet Framework}
We do not assume that the mutational  variations at the individual indices of a genomic sequence are independent (See Fig~\ref{figscheme}a). Irrespective of whether mutations are truly random~\cite{hernandez2018algorithmically}, since only certain combinations of individual mutations are viable, individual mutations across a genomic sequence replicating in the wild  appear  constrained, which is what is explicitly  modeled in our approach.

% 

Consider a set of random variables $X=\{X_i\}$, with $i \in \{1, \cdots, N\}$, each taking value from the respective sets $\Sigma_i$. Here each $X_i$ is the random variable modeling the ``outcome'' $i.e.$ the AA residue at the $i^{th}$ index of the protein sequence. A sample $x \in \prod_1^N \Sigma_i$ is an ordered $N$-tuple, which is a specific strain in this context,  consisting of a realization of each of the variables $X_i$ with the $i^{th}$ entry $x_i$ being the realization of random variable $X_i$.

We use the notation $x_{-i}$ and $x^{i,\sigma}$ to denote:
\begin{subequations}\cgather{
x_{-i} \triangleq x_1, \cdots, x_{i-1},x_{i+1},\cdots,x_N\\
x^{i,\sigma} \triangleq x_1, \cdots, x_{i-1},\sigma,x_{i+1},\cdots,x_N, \sigma \in \Sigma_i
}\end{subequations} Also, $\Dx(S)$ denotes the set of probability measures on  a set $S$, $e.g.$,  $\D$ is the set of  distributions on  $\Sigma_i$.

We note that $X$ defines a random field~\cite{vanmarcke2010random} over the index set $\{1, \cdots, N\}$. 

\begin{defn}[\tnet]
For a random field $X=\{X_i\}$ indexed by $i \in \{1, \cdots, N\}$, the \tnet is defined to be the set of predictors $\Phi=\{\qn\}$, $i.e.$, we have:
\cgather{
\qn : \prod_{j \neq i} \Sigma_j \rightarrow \D,
}  where for a sequence $x$, $\Phi_i(x_{-i}) $ estimates the distribution of $X_i$ on the set $\Sigma_i$.
\end{defn}
We use conditional inference trees as models for predictors~\cite{Hothorn06unbiasedrecursive}, although more general models are possible.





\subsection*{Biology-Aware Distance Between Sequences}
The mathematical form of our metric is not arbitrary; JS divergence is a symmetricised version of the more common KL divergence~\cite{cover} between distributions, and among  different possibilities, the \qdist  is the simplest metric such that the likelihood of a spontaneous jump (See Eq.~\eqref{fundeq} in Methods) is provably bounded above and below  by simple exponential functions of the \qdist.

\begin{defn}[\qdist: adaptive biologically meaningful dissimilarity between sequences]\label{defqdistance}
Given two sequences $x,y \in \prod_1^N\Sigma_i$, such that $x,y$ are drawn from the  populations $P,Q$  inducing the \tnet $\Phi^P,\Phi^Q$, respectively,  we define a pseudo-metric $\theta(x,y) $, as follows:
\cgather{\label{q-distance}
\theta(x,y) \triangleq \mathbf{E}_i \left (  \J^{\frac{1}{2}} \left (\qn^P(x_{-i}) , \qn^Q(y_{-i})\right ) \right )
} 
where $ \J(\cdot,\cdot)$ is the Jensen-Shannon divergence~\cite{manning1999foundations} and $\mathbf{E}_i$ indicates expectation over the indices.
\end{defn}
The square-root in the definition arises naturally from the bounds we are able to prove, and is dictated by the form of Pinsker's inequality~\cite{cover}, ensuring that   the sum of the length of successive path fragments equates the length of the path.%, making it possible to use standard  algorithms  for q-phylogeny construction.


%\subsection*{Significance Test for Population Membership}
\subsection*{Membership Degree}

For our modeling to be reliable, we need a quantitative test of how well the \tnet represents the data. Here, we formulate an explicit membership test to ascertain if individual samples may indeed be generated by the \tnet with sufficiently high probability.
%
\begin{defn}[Membership probability of a sequence]\label{defmem}
Given a population $P$ inducing the \tnet $\Phi^P$ and a sequence $x$, we can compute the membership probability of $x$:
\cgather{
\mem{x}^P \triangleq Pr(x \in P) = \prod_{j=1}^N \left ( \Phi^P_j(x_{-j}) \vert_{x_j} \right )
}
\end{defn}
$x_j$ is the $j^{th}$ entry in $x$, and is thus an element in the set $\Sigma_j$. Since we are mostly concerned with the case where $\Sigma_j$ is a finite set, $\Phi^P_j(x_{-j}) \vert_{x_j}$ is the entry in the probability mass function corresponding to the element of $\Sigma_j$ which appears at the  $j^{th}$ index in sequence $x$. 
 
We can carry out this calculation for a sequence $x$  known to be in the population $P$ as well, which allows us to define the membership degree $\M^P_x$.
\begin{defn}[Membership degree]
Let $X$ be a random field representing a population $P$, $ie.$. $X=x$ is a randomly drawn sequence from $P$. Then  the membership degree $\M^P$ is  a function of the random variable $X$: 
\cgather{
\M^P(X)  \triangleq  \prod_{j=1}^N \left ( \Phi^P_j(X_{-j}) \vert_{X_j} \right )
}Note that $\M^P$ takes values in the unit interval $[0,1]$, and the probability  $x$ is a member of the population $P$ is $\M^P(X=x)$, denoted briefly as $\mem{x}^P$ or $\mem{x}$ if $P$ is clear from context.
\end{defn}
Since $\M^P(X)$ is a random variable, we can now compute sets of sequences that better represent the population $P$, and ones that are on the fringe. We can also evaluate using a pre-specified significance-level if a particular sequence is not from the population $P$.



\subsection*{Theoretical Probability Bounds}

The \tnet framework  allows us to rigorously compute bounds on the probability of a spontaneous change of one strain to another, brought about by chance mutations. While any sequence of mutations is equally likely, the ``fitness'' of the resultant strain, or the probability that it will even result in a viable strain, or not. Thus the necessity of preserving  function  dictates that not all random changes  are viable, and the probability of observing some trajectories through the sequence space  are far greater  than others. The \tnet framework allows us to explore this constrained dynamics, as revealed by a sufficiently large set of genomic sequences.



The mathematical intuition  relating  \qdist  to the log-likelihood of spontaneous change  is similar to quantifying the  odds of  a rare biased outcome when we  toss a fair coin.
While for an unbiased coin, the odds of roughly 50\% heads is overwhelmingly likely, large deviations do happen rarely, and it turns out that the probability of such rare deviations can be explicitly quantified with existing statistical theory~\cite{varadhan2010large}.
 Generalizing to non-uniform conditional distributions inferred by the \tnet, the likelihood of a spontaneous transition  by random chance may also be similarly bounded.


We show in Theorem~\ref{thmbnd} in the supplementary text that at a significance level $\alpha$, with a sequence length $N$, the probability of spontaneous jump of sequence $x$ from population $P$ to sequence $y$ in population $Q$, $Pr(x \rightarrow y)$, is bounded by:
\cgather{\label{fundeq}
\mem{y}^Q e^{ \frac{\sqrt{8}N^2}{1-\alpha}\theta(x,y)} \geqq Pr(x \rightarrow y) \geqq \mem{y}^Q e^{-\frac{\sqrt{8}N^2}{1-\alpha}\theta(x,y)}}
where $\mem{y}^Q$ is the membership probability of strain $y$ in the target population, $N$ is the sequence length, and $\alpha$ is the statistical signifacnce level.


\subsection*{Predicting Dominant Seasonal Strains} 

Analyzing the distribution of sequences observed to circulate in the human population at the present time allows us to forecast dominant strain(s) in the next flu season as follows:

Let $\dst$ be a dominant strain in the upcoming flu season at time $t+\delta$,
where $H^t$ is the set of observed strains presently in circulation in the human population (at time $t$). We will assume that the \tnet is constructed using the sequences in teh set $H^t$, and remains unchanged upto $t+\delta$. Since this set is a function of time, the inferred \tnet also changes with time, and the induced \qdist is denoted as $\theta^{[t]}(\cdot,\cdot)$.

From the RHS bound established in Theorem~\ref{thmbnd} (See Eq.~\eqref{fundeq} above) in the supplementary text, we have:
%
\calign{
  &\ln  \frac{Pr(x \rightarrow \dsta)}{\mem{\dsta}} \geqq  -\frac{\sqrt{8}N^2}{1-\alpha}\theta^{[t]}(x,\dsta)\\
\Rightarrow &\sum_{x \in H^t} \ln  \frac{Pr(x \rightarrow \dsta)}{\mem{\dsta}}  
\geqq  \sum_{x \in H^t}-\frac{\sqrt{8}N^2}{1-\alpha}\theta^{[t]}(x,\dsta)\\
\Rightarrow  &\sum_{x\in H^t}  \theta^{[t]}(x,\dsta) - \abs{H^t}A \ln \mem{\dsta} \geqq  A \ln \frac{1}{\prod_{x \in H^t} Pr(x \rightarrow \dsta)} \intertext{where $A =\frac{1-\alpha}{\sqrt{8}N^2} $, where $N$ is the sequence length considered, and $\alpha$ is a fixed significance level. Since minimizing the LHS maximizes the lower bound on the probability of the observed strains simultaneously giving rise to $\dsta$, a dominant strain  $\dst$ may be estimated as a solution to the optimization problem:}
&\dst = \argmin_{y \in \cup_{\tau \leqq t} H^\tau} \sum_{x\in H^t}  \theta^{[t]}(x,y) - \abs{H^t}A \ln \mem{y}
}%
%
\subsection*{Measure of Pandemic Potential}
\def\ast{x_a^t}
\def\hst{x_h^{t+\delta}}

We measure the potential of an animal strain $\ast$ to spillover and become HH capable as a human strain $\hst$, via the proposed \erisk defined as follows:
\cgather{\label{erisk}
\rho(\ast) \triangleq -\frac{1}{\abs{H^t}} \sum_{x \in H^t} \theta^{[t]}(\ast,x)
}%
where as before $H^t$ is the set of human strains observed recently (we take this as strains collected within the past year), and $\theta^{[t]}$ is the \qdist induced by the \tnet computed from the sequences in $H^t$.

The intuition here is that a lower bound of $\rho(\ast)$ scales as average log-likelihood of the $\ast$ giving rise to a human strains in circulation at time $t$. Since the strains in $H^t$ are already HH capable, a high average likelihood of producing a similar strain has a high potential of being a HH cabale novel variant, which is a necessary condition of a pandemic strain. To establish the lower bound, we note that from  Theorem~\ref{thmbnd} (See Eq.~\eqref{fundeq} above) in the supplementary text, we have:
%
\cgather{
  \sum_{y \in H^t}\ln \abs{\frac{Pr(\ast \rightarrow y)}{\mem{y}}} \leqq -\frac{\sqrt{8}N^2}{1-\alpha} \abs{H^t}  \rho(\ast) \intertext{Denoting, $A =\frac{1-\alpha}{\sqrt{8}N^2} $,  $A\ln(\prod_{y \in H^t}\mem{y}) = C$, and $\langle \cdot \rangle$ as the geometric mean function, we have:}
\Rightarrow  \rho(\ast) \geqq A \ln \left (\prod_{y \in H^t}Pr(\ast \rightarrow y)\right )^{1/\abs{H^t}} + C \\
\Rightarrow \rho(\ast) \geqq A \ln \left \langle Pr(\ast \rightarrow \hst) \right \rangle + C
}%
Noting that $A,C$ are not functions of $\ast$, we conclude that a lower bound of the proposed risk measure $\rho(\cdot)$ scales with the average loglikelihood  of producing strains close to a circulating human strain at the current time. 

\subsection*{Proof of Probability Bounds}\label{sec:proof}

\begin{thm}[Probability bound]\label{thmbnd}
Given a sequence  $x$ of length $N$ that transitions  to a strain $y\in Q$, we have the following bounds at significance level $\alpha$.
\cgather{
\mem{y}^Q e^{ \frac{\sqrt{8}N^2}{1-\alpha}\theta(x,y)} \geqq Pr(x \rightarrow y) \geqq \mem{y}^Q e^{-\frac{\sqrt{8}N^2}{1-\alpha}\theta(x,y)}
  }%
  where $\mem{y}^Q$ is the membership probability of strain $y$ in the target population $Q$ (See Def.~\ref{defmem}), and $\theta(x,y)$ is the q-distance between $x,y$ (See Def.~\ref{defqdistance}).
\end{thm}
\begin{proof}
Using Sanov's theorem~\cite{cover} on large deviations, we conclude that the probability of spontaneous jump from strain $x\in P$ to strain $y\in Q$, with the possibility $P \neq Q$, is given by:
\cgather{\label{eq29}
  Pr(x\rightarrow y) =\prod_{i=1}^N \left ( \Phi^P_i(x_{-i}) \vert_{y_i} \right )
}
Writing the factors on the right hand side as:
\cgather{
 \Phi^P_i(x_{-i}) \vert_{y_i} =  \Phi^Q_i(y_{-i}) \vert_{y_i} \left (  \frac{\Phi^P_i(x_{-i}) \vert_{y_i}}{\Phi^Q_i(y_{-i}) \vert_{y_i}}  \right )
}%
we note that $\Phi^P_i(x_{-i})$, $\Phi^Q_i(y_{-i})$ are distributions on the same index $i$, and hence:
  \cgather{
\vert  \Phi^P_i(x_{-i})_{y_i} - \Phi^Q_i(y_{-i})_{y_i}\vert \leqq \sum_{y_i \in \Sigma_i} \vert  \Phi^P_i(x_{-i})_{y_i} - \Phi^Q_i(y_{-i})_{y_i}\vert 
}%
Using a standard refinement of Pinsker's inequality~\cite{fedotov2003refinements}, and the relationship of Jensen-Shannon divergence with  total variation, we get:
\cgather{
  \theta_i \geqq \frac{1}{8} \vert  \Phi^P_i(x_{-i})_{y_i} - \Phi^Q_i(y_{-i})_{y_i}\vert^2
\Rightarrow \left   \lvert  1  - \frac{\Phi^Q_i(y_{-i})_{y_i}}{\Phi^P_i(x_{-i})_{y_i}} \right \rvert \leqq \frac{1}{a_0}\sqrt{8 \theta_i}
}%
where $a_0$ is the smallest non-zero probability value of generating the entry at any index. We will see that this parameter is related to statistical significance of our bounds. First, we can formulate a lower bound as follows:
\cgather{\label{eqLB}
 \log \left  ( \prod_{i=1}^N   \frac{\Phi^P_i(x_{-i}) \vert_{y_i}}{\Phi^Q_i(y_{-i}) \vert_{y_i}}  \right )
  = \sum_i \log  \left  (  \frac{\Phi^P_i(x_{-i}) \vert_{y_i}}{\Phi^Q_i(y_{-i}) \vert_{y_i}}  \right )
\geqq \sum_i \left  ( 1- \frac{\Phi^Q_i(y_{-i})_{y_i}}{\Phi^P_i(x_{-i})_{y_i}} \right ) \geqq  \frac{\sqrt{8}}{a_0}\sum_i\theta_i^{1/2} = -\frac{\sqrt{8}N}{a_0}\theta
}%
Similarly,  the upper bound may be derived as:
\cgather{\label{eqUB}
\log \left  ( \prod_{i=1}^N   \frac{\Phi^P_i(x_{-i}) \vert_{y_i}}{\Phi^Q_i(y_{-i}) \vert_{y_i}}  \right )
  = \sum_i \log  \left  (  \frac{\Phi^P_i(x_{-i}) \vert_{y_i}}{\Phi^Q_i(y_{-i}) \vert_{y_i}}  \right ) \leqq \sum_i \left  ( \frac{\Phi^Q_i(y_{-i})_{y_i}}{\Phi^P_i(x_{-i})_{y_i}} - 1 \right ) \leqq \frac{\sqrt{8}N}{a_0}\theta
}%
Combining Eqs.~\ref{eqLB} and \ref{eqUB}, we conclude:
\cgather{
\mem{y}^Q e^{ \frac{\sqrt{8}N}{a_0}\theta} \geqq Pr(x \rightarrow y) \geqq \mem{y}^Q e^{-\frac{\sqrt{8}N}{a_0}\theta}
}%
Now, interpreting $a_0$ as the probability of generating an unlikely event below our desired threshold ($i.e.$ a ``failure''), we note that the probability of generating at least one such event is given by $1-(1-a_0)^N$. Hence if $\alpha$ is the pre-specified significance level, we have for $N >> 1 $:
\cgather{
 a_0 \approx (1 -\alpha)/N
}%
Hence, we conclude, that at significance level $\geqq \alpha$, we have the bounds:
\cgather{
\mem{y}^Q e^{ \frac{\sqrt{8}N^2}{1-\alpha}\theta} \geqq Pr(x \rightarrow y) \geqq \mem{y}^Q e^{-\frac{\sqrt{8}N^2}{1-\alpha}\theta}
  }%
\end{proof}
\begin{rem}
This bound can be rewritten in terms of the log-likelihood of the spontaneous jump and  constants independent of the  initial sequence $x$ as:
\cgather{
\left \lvert \log Pr(x \rightarrow y) -C_0 \right \vert \leqq C_1 \theta
}%
where the constants are given by:
\calign{
C_0 &= \log \mem{y}^Q \\
C_1 &= \frac{\sqrt{8} N^2}{1-\alpha}
}%
\end{rem}


\subsection*{In-silico Corroboration of \tnet{'s} Capability To Capture Biologically Meaningful Structure}
We compare the results of simulated mutational perturbations to sequences from our databases (for which we have already constructed \tnet{s}), and then use NCBI BLAST (\href{https://blast.ncbi.nlm.nih.gov/Blast.cgi}{https://blast.ncbi.nlm.nih.gov/Blast.cgi}) to identify  if  our perturbed sequences match with existing sequences in the databases (\SUPPLEMENTARY Fig.~S-\ref{figsoa}). We find that in contrast to random variations, which rapidly diverge the trajectories, the \tnet constraints tend to produce smaller variance in the trajectories, maintain a high degree of match as we extend our trajectories, and produces matches closer in time to the collection time of the  initial sequence, suggesting that the \tnet  does indeed capture realistic constraints.


\subsection*{Multivariate Regression to Understand Data Characteristics Necessary For \tnet Modeling}

We investigate the key factors that contribute to modeling a set of strains well within the \tnet framework. We carry out a multivariate regression with data diversity, the complexity of inferred \tnet and the edit distance of the WHO recommendation from the dominant strain as independent variables (See \SUPPLEMENTARY Table~S-\ref{tabreg} for definitions). Here we define data diversity as the number of clusters we have in the input set of sequences, such that any two sequences five or less mutations apart are in the same cluster. \tnet complexity is measured by the number of decision nodes in the component decision trees of the recursive forest.

We select several plausible structures of the regression equation, and in each case conclude that  data diversity has the most important and statistically significant contribution (\SUPPLEMENTARY Table~S-\ref{tabreg}).

\subsection*{Multivariate Regression to Identify Map from \qdist to Estimated IRAT scores}
We train separate General Linear Models (GLM) to estimate IRAT scores (emergence and impact) with average \qdist of a sequence of interest from a set of human strains, considering HA and NA sequences separately, using the CDC computed IRAT scores as the dependent variables. We also  include the geometric mean of the HA and NA based \qdist{s} as a potential explanatory variables. We use a standard Gaussian model family with identity link function to keep our model that maps \qdist{s}  to the IRAT scores as simple as possible (see \SUPPLEMENTARY Table~S-\ref{tabregGLMemergence}).


 



}


% \section*{Data and Software Sharing} 

% Working open-source software (requiring Python 3.x) is publicly available at \url{https://pypi.org/project/truthnet/}. All inferred \tnet models inferred is  available at \url{https://doi.org/10.5281/zenodo.7387861}.
% Accession numbers of all sequences used in this study % and acknowledgment documentation for GISAID sequences
% is provided as supplementary information (seq\_metadata.xlsx).


% \subsection*{Data Source}
% % In this study, we use sequences for the Haemagglutinnin (HA)  and Neuraminidase (NA) for Influenza A (for subtypes H1N1 and H3N2), which are key enablers of cellular entry and exit mechanisms respectively~\cite{mcauley2019influenza}.
% We use two public sequence databases: 1) National Center for Biotechnology Information (NCBI) virus~\cite{hatcher2017virus} and 2) GISAID~\cite{bogner2006global} databases. The former is a community portal for viral sequence data, aiming to increase the usability of data archived in various NCBI repositories. GISAID has a  more restricted user agreement, and use of GISAID data in an analysis requires acknowledgment of the contributions of both the submitting and the originating laboratories (Corresponding acknowledgment tables are included as supplementary information). We collected a total of $187,294$ sequences in our analysis, although not all were used due to some being duplicates (see \SUPPLEMENTARY Table~S-\ref{tabseq}).



% #############################################
% #############################################

% Bibliography
\bibliographystyle{naturemag}
\bibliography{tnet}


\clearpage                                                                      
\setcounter{figure}{0}
\renewcommand{\figurename}{Extended Data Figure}                               
\setcounter{table}{0}                                     
\renewcommand{\tablename}{Extended Data Table}                                 
%%
%#############################################
\ifFIGS
\begin{table}[t]\centering

  \mnp{\textwidth}{
\captionN{Out-performance of \enet recommendations over WHO 
for Influenza A vaccine composition}\label{tabperf}\centering
\vspace{-7pt}

\sffamily\fontsize{7}{8}\selectfont

\input{Figures/tabdata/improvement_qnet}
% \end{table*}
% \else
% \refstepcounter{table}\label{tabperf}
% \fi
% %#############################################
% %#############################################


% %#############################################
% %#############################################
% \begin{table}[!hb]\centering
}
\vskip .5em

  \mnp{\textwidth}{
\captionN{H1N1 HA Northern Hemisphere}\label{tabrec0}
\vspace{-10pt}

\sffamily\fontsize{7}{8}\selectfont

\input{Figures/tabdata/north_h1n1_ha.tex}
\flushleft

\fontsize{7}{7}\selectfont
$^\star$ Dominant strain is calculated as the one closest to the centroid in the strain space that year in the edit distance metric

% \end{table}

% %#############################################
% %#############################################

% \begin{table}[!hb]\centering
}
\vskip .5em
\mnp{\textwidth}{
  
\captionN{H1N1 HA Southern Hemisphere}\label{tabrec1}
\vspace{-10pt}

\sffamily\fontsize{7}{8}\selectfont

\input{Figures/tabdata/south_h1n1_ha.tex}
\flushleft

\fontsize{8}{8}\selectfont
$^\star$ Dominant strain is calculated as the one closest to the centroid in the strain space that year in the edit distance metric
}
\end{table}
 \else
 \refstepcounter{table}\label{tabperf}
 \fi
% %#############################################
% %#############################################
\begin{table}[!ht]\centering
\captionN{H3N2 HA Northern Hemisphere}\label{tabrec2}

\sffamily\fontsize{7}{8}\selectfont

\input{Figures/tabdata/north_h3n2_ha.tex}
\flushleft

\fontsize{7}{7}\selectfont
$^\star$ Dominant strain is calculated as the one closest to the centroid in the strain space that year in the edit distance metric
\end{table}
%#############################################
%#############################################

\begin{table}[!ht]\centering
\captionN{H3N2 HA Southern Hemisphere}\label{tabrec3}

\sffamily\fontsize{7}{8}\selectfont

\input{Figures/tabdata/south_h3n2_ha.tex}
\flushleft

\fontsize{7}{7}\selectfont
$^\star$ Dominant strain is calculated as the one closest to the centroid in the strain space that year in the edit distance metric
\end{table}
%#############################################
%#############################################

\clearpage

%#############################################
%#############################################
\ifFIGS
\begin{figure*}[!ht]
  \centering
  \tikzexternalenable
    \tikzsetnextfilename{sequence}
\vspace{-5pt}
 %\tikzXtrue
  \iftikzX
  \input{Figures/sequence_}  
  \vspace{0pt}   
  
  \else
  \includegraphics[width=0.87\textwidth]{Figures/External/sequence.pdf}  \vspace{-5pt}   

  \fi
  
\vspace{0pt}

\captionN{\textbf{Sequence comparisons.} Comparing the \enet  (ENT) and the WHO recommendation (WHO), and the observed dominant strain (DOM), we note that the correct \enet  predictions tend to be within the RBD, both for H1N1 and H3N2 for HA (panel a shows one example). Additionally, by comparing the type, side chain area, and the accessible side chain area, we note that DOM and ENT are often close in important chemical properties, while WHO deviations are  not (panel b-f). Panels g-i show the localization of the deviations in the molecular structure of HA, where we note that the changes are most frequent in the HA1 sub-unit (the globular head), and around residues and structures that have been commonly implicated in receptor binding interactions $e.g$ the $\approx 200$ loop, the $\approx 220$ loop and the $\approx 180$-helix~\cite{tzarum2015structure,lazniewski2018structural,garcia2015dynamic}.}\label{figseq}
\end{figure*}
\else
\refstepcounter{figure}\label{figseq}
\fi
%#############################################
%#############################################
%#############################################
\ifFIGS

\begin{table}[!ht]\centering
\captionN{Influenza A Strains Evaluated by IRAT and Corresponding \enet Computed Risk Scores}\label{irattab}

\sffamily\fontsize{7}{8}\selectfont

\input{Figures/tabdata/irat_predictions}
\flushleft

 \fontsize{8}{8}\selectfont
 $^{\star\star}$  \enet constructed using all human strains that match the HA sub-type, $e.g.$, H5Nx for H5N6.\\
 $^{\star\star\star}$ distance estaimated averaging over those obtained by considering all \enet{s} from other subtypes.
\end{table}
\else
\refstepcounter{table}\label{irattab}
\fi
% #############################################

%#############################################



\ifFIGS

\begin{table}[!ht]\centering
\captionN{Count of identified strains above estimated emergence risk threshold}\label{riskytab}

\sffamily\fontsize{7}{8}\selectfont

\input{Figures/tabdata/riskycount}
\end{table}
\else
\refstepcounter{table}\label{riskytab}
\fi
% #############################################
%#############################################
\ifFIGS

\begin{table}[!ht]\centering
\captionN{Influenza A Strains Evaluated by IRAT and Corresponding \enet Computed Risk Scores}\label{highrisktab}

\bf\sffamily\fontsize{7}{7}\selectfont

\input{Figures/tabdata/highrisk35}
\end{table}
\else
\refstepcounter{table}\label{highrisktab}
\fi
% #############################################

\ifFIGS

\begin{figure}[!ht]
  \tikzexternalenable
  \tikzsetnextfilename{riskyseq}
  \centering
 %\tikzXtrue
 
  
  \iftikzX  
  \input{Figures/sequence_risky}
 \else
  \includegraphics[width=.95\textwidth]{Figures/External/riskyseq}
  \fi 
  \vspace{-18pt}
  
\captionN{HA sequence comparison  with dominant human strains (DOM\_HUMAN H1N1, H3N2)  with \enet estimated top 5 risky strains (2020-2022 April) along with the teh most risky H9N2 strain (A/mink/China/chick embryo/2020), showing substantial differences from the circulating strains both in and out of the RBD. }\label{figriskyseq}
\end{figure}
\else
\refstepcounter{figure}\label{figriskyseq}
\fi

% #############################################



\clearpage



\section*{Supplementary Figures \& Tables}

%\input{SIfig.tex}

\end{document}
% LocalWords:  Neuraminidase subtype

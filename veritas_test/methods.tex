% methods addendum.tex
\def\qnet{TruthNet\xspace}
\def\vrts[VeRITAS\xspace}

\def\TITLE{[Ve]rification of [R]esponse [I]ntergrity in [A]dversarial [S]urveys via generative modeling of cross-talk patterns and emergent complexity bounds of natural responses}

\section{Algorithm \vrts}



\begin{notn}
  A \textit{response vector} is an ordered sequence of responses to a subset of items from a fixed item bank. The notions of ``items'' and ``item banks'' are assumed to be what they are generally understood to be. A response vector can have missing respones.
  
$\Phi^M$ is used to denote a \qnet model. As a function,   $\Phi^{M}(x)$ specifies  a set of  probability distributions, one  for each of the  items in the item bank (or the set of items which were used to construct the \qnet), given a response vector $x$. $M$ denotes the population for which the \qnet was inferred. We will use the notation $\Phi^0(x,i)$ to denote the model for a control population, and $\Phi^+(x,i)$ for the cases, $e.g.$, subjects diagnosed with PTSD.

Two more concepts are $\omega(x)=Pr(x \rightarrow x)$, also called the response persistence, and $\delta(x)$ known as the response defect. $\omega$ can be also seen as a quantificiation to how well $x$ corresponds to the inferred model, and the likelihood of it being generated by the model. $\delta$ is the expected probability of deviation from the most likely response generated by the model.


\end{notn}

